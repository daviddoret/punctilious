\boldsymbol\mathsf{Proposition}} \boldsymbol\mathsf{2.4.1}} (\mathnormal{P}_{31})\mathsf{: }\left(\left(\left(\left(\mathbf{n}_{3} \mathit{is-a} \mathit{natural-number}\right) \mathit{\land} \left(\mathbf{m}_{1} \mathit{is-a} \mathit{natural-number}\right)\right) \mathit{\land} \left(\mathbf{n}_{3} \mathit{\neq} \mathbf{m}_{1}\right)\right) \mathit{\implies} \left(\left(\mathbf{n}_{3}\right)\mathit{++} \mathit{\neq} \left(\mathbf{m}_{1}\right)\mathit{++}\right)\right)\mathsf{.}\mathsf{ }\boldsymbol\mathsf{Proof}}\mathsf{: }\left\ulcorner\text{\sffamily{\itshape{Different natural numbers must have different successors; i.e., if n, m are natural numbers and n  m, then n++  m++. Equivalently, if n++ = m++, then we must have n = m.}}}\right\ulcorner\mathsf{ is postulated by }\boldsymbol\mathsf{axiom}} \boldsymbol\mathsf{2.4}} (\mathnormal{A}_{8})\mathsf{. }\left(\left(\left(\left(\mathbf{n}_{3} \mathit{is-a} \mathit{natural-number}\right) \mathit{\land} \left(\mathbf{m}_{1} \mathit{is-a} \mathit{natural-number}\right)\right) \mathit{\land} \left(\mathbf{n}_{3} \mathit{\neq} \mathbf{m}_{1}\right)\right) \mathit{\implies} \left(\left(\mathbf{n}_{3}\right)\mathit{++} \mathit{\neq} \left(\mathbf{m}_{1}\right)\mathit{++}\right)\right)\mathsf{ is an interpretation of that axiom.}\mathsf{ Therefore, by the }\mathit{axiom-interpretation}\mathsf{ inference rule, it follows that }\left(\left(\left(\left(\mathbf{n}_{3} \mathit{is-a} \mathit{natural-number}\right) \mathit{\land} \left(\mathbf{m}_{1} \mathit{is-a} \mathit{natural-number}\right)\right) \mathit{\land} \left(\mathbf{n}_{3} \mathit{\neq} \mathbf{m}_{1}\right)\right) \mathit{\implies} \left(\left(\mathbf{n}_{3}\right)\mathit{++} \mathit{\neq} \left(\mathbf{m}_{1}\right)\mathit{++}\right)\right)\mathsf{. }\mathsf{\qed}